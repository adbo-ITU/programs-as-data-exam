\section{List–C: Tabeller på hoben}

\subsection{Abstrakt syntaks}

\begin{minted}{diff}
--- a/2021-jan/opgave3/ListC/Absyn.fs
+++ b/2021-jan/opgave3/ListC/Absyn.fs
@@ -24,2 +24,3 @@ and expr =
   | Prim2 of string * expr * expr    (* Binary primitive operator   *)
+  | Prim3 of string * expr * expr * expr
   | Andalso of expr * expr           (* Sequential and              *)
\end{minted}

\begin{minted}{fsharp}
> Stmt (Expr (Prim3 ("updateTable", Access (AccVar "t"), CstI 0, CstI 42)));;
val it : stmtordec =
  Stmt (Expr (Prim3 ("updateTable", Access (AccVar "t"), CstI 0, CstI 42)))
\end{minted}

\subsection{Lexer og parser}

\begin{minted}{diff}
--- a/2021-jan/opgave3/ListC/CLex.fsl
+++ b/2021-jan/opgave3/ListC/CLex.fsl
@@ -34,2 +34,6 @@ let keyword s =
     | "setcdr"  -> SETCDR     
+    | "createTable" -> CREATETABLE
+    | "updateTable" -> UPDATETABLE
+    | "indexTable" -> INDEXTABLE
+    | "printTable" -> PRINTTABLE
     | "true"    -> CSTBOOL 1

--- a/2021-jan/opgave3/ListC/CPar.fsy
+++ b/2021-jan/opgave3/ListC/CPar.fsy
@@ -17,2 +17,3 @@ let nl = CstI 10
 %token CHAR ELSE IF INT NULL PRINT PRINTLN RETURN VOID WHILE
+%token CREATETABLE UPDATETABLE INDEXTABLE PRINTTABLE
 %token NIL CONS CAR CDR DYNAMIC SETCAR SETCDR
@@ -148,2 +149,6 @@ AtExprNotAccess:
   | SETCDR LPAR Expr COMMA Expr RPAR        { Prim2("setcdr",$3,$5) }
+  | CREATETABLE LPAR Expr RPAR              { Prim1("createTable", $3)         }
+  | UPDATETABLE LPAR Expr COMMA Expr COMMA Expr RPAR { Prim3("updateTable", $3, $5, $7) }
+  | INDEXTABLE LPAR Expr COMMA Expr RPAR    { Prim2("indexTable", $3, $5)      }
+  | PRINTTABLE LPAR Expr RPAR               { Prim1("printTable", $3)          }
 ;
\end{minted}

\begin{minted}{fsharp}
$ mono listcc.exe exam10.lc
List-C compiler v 1.0.0.0 of 2012-02-13
Compiling exam10.lc to exam10.out
Parsed: Prog
  [Fundec
     (None, "main", [],
      Block
        [Dec (TypD, "t");
         Stmt (Expr (Assign (AccVar "t", Prim1 ("createTable", CstI 3))));
         Stmt
           (Expr (Prim3 ("updateTable", Access (AccVar "t"), CstI 0, CstI 42)));
         Stmt
           (Expr (Prim3 ("updateTable", Access (AccVar "t"), CstI 1, CstI 43)));
         Stmt
           (Expr (Prim3 ("updateTable", Access (AccVar "t"), CstI 2, CstI 44)));
         Stmt
           (Expr
              (Prim1
                 ("printi", Prim2 ("indexTable", Access (AccVar "t"), CstI 0))));
         Stmt
           (Expr
              (Prim1
                 ("printi", Prim2 ("indexTable", Access (AccVar "t"), CstI 1))));
         Stmt
           (Expr
              (Prim1
                 ("printi", Prim2 ("indexTable", Access (AccVar "t"), CstI 2))));
         Stmt (Expr (Prim1 ("printTable", Access (AccVar "t"))))])]
ERROR: unknown primitive 1
\end{minted}

\subsection{Implementering af bytekodefortolker}

\begin{minted}{diff}
--- a/2021-jan/opgave3/ListC/ListVM/ListVM/listmachine.c
+++ b/2021-jan/opgave3/ListC/ListVM/ListVM/listmachine.c
@@ -81,7 +81,7 @@ created when allocating all but the last word of a free block.
   #define PPCOMP "Not compiled with gcc."
 #endif
 
-#if _WIN64 || __x86_64__ || __ppc64__
+#if _WIN64 || __x86_64__ || __ppc64__ || __aarch64__
   #define ENV64
   #define PPARCH "64 bit architecture."
 #else
@@ -151,6 +151,7 @@ word* readfile(char* filename);
 #endif
 
 #define CONSTAG 0
+#define TABLETAG 1
 
 // Heap size in words
 
@@ -195,6 +196,10 @@ word *freelist;
 #define CDR 29
 #define SETCAR 30
 #define SETCDR 31
+#define CREATETABLE 32
+#define UPDATETABLE 33
+#define INDEXTABLE 34
+#define PRINTTABLE 35
 
 #define STACKSIZE 1000
 
@@ -237,6 +242,10 @@ void printInstruction(word p[], word pc) {
   case CDR:    printf("CDR"); break;
   case SETCAR: printf("SETCAR"); break;
   case SETCDR: printf("SETCDR"); break;
+  case CREATETABLE: printf("CREATETABLE"); break;
+  case UPDATETABLE: printf("UPDATETABLE"); break;
+  case INDEXTABLE: printf("INDEXTABLE"); break;
+  case PRINTTABLE: printf("PRINTTABLE"); break;
   default:     printf("<unknown>"); break;
   }
 }
@@ -379,6 +388,63 @@ int execcode(word p[], word s[], word iargs[], int iargc, int /* boolean */ trac
       word* p = (word*)s[sp];
       p[2] = v;
     } break;
+
+    case CREATETABLE: {
+      word N = Untag((word) s[sp--]);
+
+      if (N < 0) {
+        printf("Cannot create table with negative size\n");
+        return -1;
+      }
+
+      word* t = allocate(TABLETAG, N + 1, s, sp);
+
+      t[1] = N;
+      for (int i = 2; i <= N + 1; i++) {
+        t[i] = 0;
+      }
+
+      s[++sp] = (word) t;
+    } break;
+
+    case UPDATETABLE: {
+      word v = Untag((word) s[sp--]);
+      word i = Untag((word) s[sp--]);
+      word* t = (word*) s[sp];
+      word N = t[1];
+
+      if (i < 0 || i >= N) {
+        printf("Index %lld out of bounds\n", i);
+        return -1;
+      }
+
+      t[i + 2] = v;
+    } break;
+
+    case INDEXTABLE: {
+      word i = Untag((word) s[sp--]);
+      word* t = (word*) s[sp--];
+      word N = t[1];
+
+      if (i < 0 || i >= N) {
+        printf("Index %lld out of bounds\n", i);
+        return -1;
+      }
+
+      s[++sp] = Tag(t[i + 2]);
+    } break;
+    case PRINTTABLE: {
+      word* t = (word*) s[sp];
+      word N = t[1];
+
+      printf("[");
+      for (int i = 2; i < N + 1; i++)
+        printf("%lld, ", t[i]);
+      if (N > 0)
+        printf("%lld", t[N + 1]);
+      printf("]\n");
+    } break;
+
     default:
       printf("Illegal instruction " WORD_FMT " at address " WORD_FMT "\n",
 	     p[pc - 1], pc - 1);
\end{minted}

\subsection{Udvidelse af oversætteren}

\begin{minted}{diff}
--- a/2021-jan/opgave3/ListC/Comp.fs
+++ b/2021-jan/opgave3/ListC/Comp.fs
@@ -180,7 +180,9 @@ and cExpr (e : expr) (varEnv : varEnv) (funEnv : funEnv) : instr list =
          | "printc" -> [PRINTC]
          | "car"    -> [CAR]
          | "cdr"    -> [CDR]
+         | "createTable" -> [CREATETABLE]
+         | "printTable" -> [PRINTTABLE]
          | _        -> raise (Failure "unknown primitive 1"))
     | Prim2(ope, e1, e2) ->
       cExpr e1 varEnv funEnv
       @ cExpr e2 varEnv funEnv
@@ -199,7 +201,16 @@ and cExpr (e : expr) (varEnv : varEnv) (funEnv : funEnv) : instr list =
          | "cons"   -> [CONS]
          | "setcar" -> [SETCAR]
          | "setcdr" -> [SETCDR]
+         | "indexTable" -> [INDEXTABLE]
          | _        -> raise (Failure "unknown primitive 2"))
+    | Prim3(ope, e1, e2, e3) ->
+      cExpr e1 varEnv funEnv
+      @ cExpr e2 varEnv funEnv
+      @ cExpr e3 varEnv funEnv
+      @ (match ope with
+         | "updateTable" -> [UPDATETABLE]
+         | _        -> raise (Failure "unknown primitive 3"))
     | Andalso(e1, e2) ->
       let labend   = newLabel()
       let labfalse = newLabel()

--- a/2021-jan/opgave3/ListC/Machine.fs
+++ b/2021-jan/opgave3/ListC/Machine.fs
@@ -47,6 +47,12 @@ type instr =
   | CDR                                (* get second field of cons cell   *)
   | SETCAR                             (* set first field of cons cell    *)
   | SETCDR                             (* set second field of cons cell   *)
+  | CREATETABLE
+  | UPDATETABLE
+  | INDEXTABLE
+  | PRINTTABLE
 
 (* Generate new distinct labels *)
 
@@ -103,6 +109,10 @@ let CODECAR    = 28;
 let CODECDR    = 29;
 let CODESETCAR = 30;
 let CODESETCDR = 31;
+let CODECREATETABLE = 32
+let CODEUPDATETABLE = 33
+let CODEINDEXTABLE = 34
+let CODEPRINTTABLE = 35
 
 (* Bytecode emission, first pass: build environment that maps 
    each label to an integer address in the bytecode.
@@ -143,6 +153,10 @@ let makelabenv (addr, labenv) instr =
     | CDR            -> (addr+1, labenv)
     | SETCAR         -> (addr+1, labenv)
     | SETCDR         -> (addr+1, labenv)
+    | CREATETABLE    -> (addr+1, labenv)
+    | UPDATETABLE    -> (addr+1, labenv)
+    | INDEXTABLE     -> (addr+1, labenv)
+    | PRINTTABLE     -> (addr+1, labenv)
 
 (* Bytecode emission, second pass: output bytecode as integers *)
 
@@ -181,6 +195,10 @@ let rec emitints getlab instr ints =
     | CDR            -> CODECDR    :: ints
     | SETCAR         -> CODESETCAR :: ints
     | SETCDR         -> CODESETCDR :: ints
+    | CREATETABLE    -> CODECREATETABLE :: ints
+    | UPDATETABLE    -> CODEUPDATETABLE :: ints
+    | INDEXTABLE     -> CODEINDEXTABLE :: ints
+    | PRINTTABLE     -> CODEPRINTTABLE :: ints
 
 
 (* Convert instruction list to int list in two passes:
\end{minted}

\subsection{Uddata}

Ved kørsel af programmet fås:

\begin{minted}{shell}
$ mono listcc.exe exam10.lc
List-C compiler v 1.0.0.0 of 2012-02-13
Compiling exam10.lc to exam10.out

$ gcc -Wall -g -o listmachine ./ListVM/ListVM/listmachine.c

$ ./listmachine exam10.out
42 43 44 [42, 43, 44]

Used 0 cpu milli-seconds
\end{minted}
