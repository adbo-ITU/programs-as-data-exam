\section{List-C: Stack}

\subsection{Lexer og parser}

Ændringerne til lexeren og parseren er som følgende:

\begin{minted}{diff}
diff CLex.fsl
@@ -37,2 +37,6 @@ let keyword s =
     | "while"       -> WHILE         
+    | "createStack" -> CREATESTACK
+    | "pushStack"   -> PUSHSTACK
+    | "popStack"    -> POPSTACK
+    | "printStack"  -> PRINTSTACK
     | _             -> NAME s

diff CPar.fsy
@@ -17,2 +17,3 @@ let nl = CstI 10
 %token CHAR ELSE IF INT NULL PRINT PRINTLN RETURN VOID WHILE
+%token CREATESTACK PUSHSTACK POPSTACK PRINTSTACK
 %token NIL CONS CAR CDR DYNAMIC SETCAR SETCDR
@@ -140,2 +141,3 @@ AtExprNotAccess:
     Const                               { CstI $1               }
+  | StackOperations                     { $1                    }
   | LPAR ExprNotAccess RPAR             { $2                    } 
@@ -150,2 +152,9 @@ AtExprNotAccess:
 
+StackOperations:
+      CREATESTACK LPAR Expr RPAR          { Prim1("createStack", $3)   }
+    | PUSHSTACK LPAR Expr COMMA Expr RPAR { Prim2("pushStack", $3, $5) }
+    | POPSTACK LPAR Expr RPAR             { Prim1("popStack", $3)      }
+    | PRINTSTACK LPAR Expr RPAR           { Prim1("printStack", $3)    }
+;

 Access:
\end{minted}

4 tokens er tilføjet på linje 13. Linje 4-7 viser lexerændringerne, der oversætter keywords til tokens. Linje 17 viser en ny grammatikregel \texttt{StackOperations}, som defineres på linje 21-26. Det er ikke nødvendigt at dedikere en separat blok til \texttt{StackOperations} --- man kunne sagtens have indsat de 4 nye grammatikregler under \texttt{AtExprNotAccess}, men jeg flyttede det ud, for at gøre koden mere overskuelig.
\\\\
Derefter har jeg ændret \texttt{ListCC.fs}, så det abstrakte syntakstræ bliver skrevet til skærmen:

\begin{minted}{diff}
diff ListCC.fs
@@ -13,3 +13,6 @@ let _ = if args.Length > 1 then
            printf "Compiling %s to %s\n" source target;
-           try ignore (Comp.compileToFile (Parse.fromFile source) target)
+           try
+                let s = Parse.fromFile source
+                printfn "Parsed: %A" s
+                ignore (Comp.compileToFile (s) target)
            with Failure msg -> printf "ERROR: %s\n" msg
\end{minted}

Efter \texttt{listcc.exe} er bygget med de nye ændringer, parses \texttt{stack.lc} således:

\begin{minted}{fsharp}
$ mono listcc.exe stack.lc
List-C compiler v 1.0.0.0 of 2012-02-13
Compiling stack.lc to stack.out
Parsed: Prog
  [Fundec
     (None, "main", [],
      Block
        [Dec (TypD, "s");
         Stmt (Expr (Assign (AccVar "s", Prim1 ("createStack", CstI 3))));
         Stmt (Expr (Prim2 ("pushStack", Access (AccVar "s"), CstI 42)));
         Stmt (Expr (Prim2 ("pushStack", Access (AccVar "s"), CstI 43)));
         Stmt (Expr (Prim1 ("printStack", Access (AccVar "s"))));
         Stmt (Expr (Prim1 ("printi", Prim1 ("popStack", Access (AccVar "s")))));
         Stmt (Expr (Prim1 ("printi", Prim1 ("popStack", Access (AccVar "s")))));
         Stmt (Expr (Prim1 ("printStack", Access (AccVar "s"))))])]
ERROR: unknown primitive 1
\end{minted}

Bemærk at fejlen \texttt{ERROR: unknown primitive 1} er forventet, idet compileren ikke er udvidet med logikken til at håndtere stack-primitiver endnu.
\\\\
Derudover differentierer linje 14 sig fra opgavens ``forventede'' uddata, idet det givne eksempel på uddata har en variabel med navnet \texttt{t} i stedet for \texttt{s}. Jeg antager, at dette er en fejl i opgaveformuleringen.

\subsection{Implementering af stak i bytekodefortolkeren}

Kodeændringerne beskrives først (i rækkefølgen, ændringerne vises), hvorefter de vises:

\begin{itemize}
    \item Linje 5: En lille ændring, så bytekodefortolkeren virker på macOS med ARM-processorer.
    \item Linje 9: Et tag, der bruges i headeren, når et objekt allokeres på hoben. Det bruges til at vide, at typen på det allokerede objekt er en stak.
    \item Linje 13-16: Bytekodeinstruktionernes tal forbindes med deres navn.
    \item Linje 20-23: Kode til at udskrive bytekodeinstruktionerne.
    \item Linje 27-32: En hjælperfunktion til at udskrive værdier afhængigt af deres tag (heltal og referencer).
    \item Linje 37-53: Implementering af \texttt{createStack}. Der sørges for at $N$ fra stakken bliver untagget --- og hvis $N$ er negativt, fejler programmet. Et objekt med størrelsen $N + 2$ bliver allokeret, da der skal være plads til $N$ elementer samt 2 pladser til $N$ og \texttt{top}. Headeren bliver tilføjet af \texttt{allocate}. Derefter initialiseres stakken med $N$, \texttt{top} som 0, og resten af pladserne udfyldes med et 0 (tagget som heltal). Adressen tilføjes til programstakken.
    \item Linje 54-68: Implementering af \texttt{pushStack}. Værdien, der skal skubbes på stakken, læses (uden at untagge den, så typen bliver bibeholdt), og selve stakkens adresse læses også. Derfra læses $N$ og \texttt{top} fra hoben. Hvis stakken er fuld, fejler programmet. Stakkens adresse efterlades på programstakken.
    \item Linje 69-80: Implementering af \texttt{popStack}. Stakkens adresse læses fra programstakken. Derfra læses \texttt{top} fra hoben. Hvis stakken er tom, fejler programmet. Værdien af \texttt{top} dekrementeres med 1, og den poppede værdi efterlades på programstakken.
    \item Linje 81-90: Implementering af \texttt{printStack}. Stakkens adresse læses fra programstakken (uden at poppe den). Derfra læses og udskrives $N$ og \texttt{top} fra hoben. Ved brug af en løkke og hjælpefunktionen \texttt{printWord} udskrives alle elementer i stakken type-afhængigt.
    \item \texttt{Machine.fs} er blot udvidet med de nye instruktioner og bytekode-tal.
\end{itemize}

\begin{minted}{diff}
diff listmachine.c
@@ -83,3 +83,3 @@ created when allocating all but the last word of a free block.
 
-#if _WIN64 || __x86_64__ || __ppc64__
+#if _WIN64 || __x86_64__ || __ppc64__ || __aarch64__
   #define ENV64
@@ -153,2 +153,3 @@ word* readfile(char* filename);
 #define CONSTAG 0
+#define STACKTAG 1
 
@@ -197,2 +198,6 @@ word *freelist;
 #define SETCDR 31
+#define CREATESTACK 32
+#define PUSHSTACK   33
+#define POPSTACK    34
+#define PRINTSTACK  35
 
@@ -239,2 +244,6 @@ void printInstruction(word p[], word pc) {
   case SETCDR: printf("SETCDR"); break;
+  case CREATESTACK: printf("CREATESTACK"); break;
+  case PUSHSTACK: printf("PUSHSTACK"); break;
+  case POPSTACK: printf("POPSTACK"); break;
+  case PRINTSTACK: printf("PRINTSTACK"); break;
   default:     printf("<unknown>"); break;
@@ -243,2 +252,9 @@ void printInstruction(word p[], word pc) {
 
+void printWord(word w) {
+  if (IsInt(w))
+    printf(WORD_FMT " ", Untag(w));
+  else
+    printf("#" WORD_FMT " ", w);
+}
+
 // Print current stack (marking heap references by #) and current instruction
@@ -381,2 +397,56 @@ int execcode(word p[], word s[], word iargs[], int iargc, int /* boolean */ trac
     } break;
+    case CREATESTACK: {
+      word N = Untag(s[sp--]);
+
+      if (N < 0) {
+        printf("Cannot create stack with negative size\n");
+        return -1;
+      }
+
+      word* stack = allocate(STACKTAG, N + 2, s, sp);
+
+      stack[1] = N;
+      stack[2] = 0;
+      for (int i = 3; i <= N + 2; i++)
+        stack[i] = Tag(0);
+    
+      s[++sp] = (word) stack;
+    } break;
+    case PUSHSTACK: {
+      word v = s[sp--];
+      word* stack = (word*) s[sp];
+
+      word N = stack[1];
+      word top = stack[2];
+
+      if (top >= N) {
+        printf("Stack is full - cannot push more items\n");
+        return -1;
+      }
+
+      stack[2] = top + 1;
+      stack[top + 3] = v;
+    } break;
+    case POPSTACK: {
+      word* stack = (word*) s[sp--];
+      word top = stack[2];
+
+      if (top <= 0) {
+        printf("Stack is empty - cannot pop\n");
+        return -1;
+      }
+
+      stack[2] = top - 1;
+      s[++sp] = stack[top + 3 - 1];
+    } break;
+    case PRINTSTACK: {
+      word* stack = (word*) s[sp];
+      word N = stack[1];
+      word top = stack[2];
+
+      printf("STACK(" WORD_FMT ", " WORD_FMT ")[ ", N, top);
+      for (int i = 3; i < top + 3; i++)
+        printWord(stack[i]);
+      printf("]\n");
+    } break;
     default:

diff Machine.fs
@@ -49,2 +49,6 @@ type instr =
   | SETCDR                             (* set second field of cons cell   *)
+  | CREATESTACK
+  | PUSHSTACK
+  | POPSTACK
+  | PRINTSTACK
 
@@ -105,2 +109,6 @@ let CODESETCAR = 30;
 let CODESETCDR = 31;
+let CODECREATESTACK = 32;
+let CODEPUSHSTACK   = 33;
+let CODEPOPSTACK    = 34;
+let CODEPRINTSTACK  = 35;
 
@@ -145,2 +153,6 @@ let makelabenv (addr, labenv) instr =
     | SETCDR         -> (addr+1, labenv)
+    | CREATESTACK    -> (addr+1, labenv)
+    | PUSHSTACK      -> (addr+1, labenv)
+    | POPSTACK       -> (addr+1, labenv)
+    | PRINTSTACK     -> (addr+1, labenv)
 
@@ -183,2 +195,6 @@ let rec emitints getlab instr ints =
     | SETCDR         -> CODESETCDR :: ints
+    | CREATESTACK    -> CODECREATESTACK :: ints
+    | PUSHSTACK      -> CODEPUSHSTACK   :: ints
+    | POPSTACK       -> CODEPOPSTACK    :: ints
+    | PRINTSTACK     -> CODEPRINTSTACK  :: ints
\end{minted}

\subsection{Implementering af stak i oversætteren}

Kodeændringen er vist forneden. 1-arguments-primitivet er udvidet med \texttt{createStack}, \texttt{popStack}, og \texttt{printStack}. 2-arguments-primitivet er udvidet med \texttt{pushStack}. De foregående værdier på stakken, som de nye instruktioner har brug for, tilføjes i forvejen af \texttt{cExpr}.

\begin{minted}{diff}
diff Comp.fs
@@ -182,2 +182,5 @@ and cExpr (e : expr) (varEnv : varEnv) (funEnv : funEnv) : instr list =
          | "cdr"    -> [CDR]
+         | "createStack" -> [CREATESTACK]
+         | "popStack"    -> [POPSTACK]
+         | "printStack"  -> [PRINTSTACK]
          | _        -> raise (Failure "unknown primitive 1"))
@@ -201,2 +204,3 @@ and cExpr (e : expr) (varEnv : varEnv) (funEnv : funEnv) : instr list =
          | "setcdr" -> [SETCDR]
+         | "pushStack" -> [PUSHSTACK]
          | _        -> raise (Failure "unknown primitive 2"))
\end{minted}

Ved at køre bytekodefortolkeren op mod \texttt{stack.lc} (hvor \texttt{stack.out} kommer fra de nye ændringer i oversætteren) fås:

\begin{minted}{shell}
$ ./listmachine ./stack.out
STACK(3, 2)[ 42 43 ]
43 42 STACK(3, 0)[ ]

Used 0 cpu milli-seconds
\end{minted}

Opgavebeskrivelsen siger, at stakken skal udskrives som \texttt{STACK(3, 2)[ 42 43 ]}, men det givne eksempel viser \texttt{STACK(3, 2)=[ 42 43 ]} (bemærk det tilføjede lighedstegn). Jeg har antaget, at jeg skal følge opgavebeskrivelsen, og jeg har derfor ikke printet et lighedstegn.

\subsection{Test af stakken}

De vigtigste ting at teste er, at stakken fungerer som forventet i de mest sædvanlige tilfælde, samt at den fungerer i edge cases. Fordi \texttt{stack.lc} ikke tester edge cases, vil jeg fokusere på dem.

\begin{enumerate}
    \item \texttt{createStack} bør virke, når størrelsen er 0:
        \begin{minted}[autogobble]{c}
            void main() {
              printStack(createStack(0));
            }
        \end{minted}
        Uddata bekræfter, at testen består:
        \begin{minted}[autogobble]{shell}
            $ ./listmachine test-createStack-empty.out
            STACK(0, 0)[ ]

            Used 0 cpu milli-seconds
        \end{minted}
    \item \texttt{createStack} bør fejle, når størrelsen er negativ:
        \begin{minted}[autogobble]{c}
            void main() {
              printStack(createStack(-1));
            }
        \end{minted}
        Uddata viser, at programmet fejler (som forventet):
        \begin{minted}[autogobble]{shell}
            $ ./listmachine test-createStack-negative.out
            Cannot create stack with negative size

            Used 0 cpu milli-seconds
        \end{minted}
    \item \texttt{pushStack} bør bevare typen på den givne værdi --- så heltal opbevares som heltal, og referencer til hoben opbevares som referencer:
        \begin{minted}[autogobble]{c}
            void main() {
              dynamic s;
              s = createStack(3);
            
              pushStack(s, s);
              pushStack(s, 100);
            
              printStack(s);
            
              print popStack(s);
              print popStack(s);
            }
        \end{minted}
        Uddata viser, at bytekodefortolkeren differentierer mellem referencer til hoben og skalarværdier internt, samt at værdierne korrekt poppes ud:
        \begin{minted}[autogobble]{shell}
            $ ./listmachine test-pushStack-types.out
            STACK(3, 2)[ #5813341184 100 ]
            100 5813341184
            Used 0 cpu milli-seconds
        \end{minted}
    
    \item \texttt{pushStack} bør fejle, når der pushes til en fuld stak:
        \begin{minted}[autogobble]{c}
            void main() {
              dynamic s;
              s = createStack(1);
              pushStack(s, 1);
              printStack(s);
              pushStack(s, 2);
              printStack(s);
            }
        \end{minted}
        Uddata viser, at programmet fejler (som forventet):
        \begin{minted}[autogobble]{shell}
            $ ./listmachine test-pushStack-pushfull.out
            STACK(1, 1)[ 1 ]
            Stack is full - cannot push more items
            
            Used 0 cpu milli-seconds
        \end{minted}
\end{enumerate}
