\section{Micro–C: Print Stack}

\subsection{Lexer og parser}

Jeg har ændret lexeren og parseren som følgende:

\begin{minted}{diff}
--- a/2022-jan/opgave-2/MicroC/Absyn.fs
+++ b/2022-jan/opgave-2/MicroC/Absyn.fs
@@ -35,6 +35,7 @@ and stmt =
   | Expr of expr                     (* Expression statement   e;   *)
   | Return of expr option            (* Return from method          *)
   | Block of stmtordec list          (* Block: grouping and scope   *)
+  | PrintStack of expr
                                                                    
 and stmtordec =                                                    
   | Dec of typ * string              (* Local variable declaration  *)

--- a/2022-jan/opgave-2/MicroC/CLex.fsl
+++ b/2022-jan/opgave-2/MicroC/CLex.fsl
@@ -25,6 +25,7 @@ let keyword s =
     | "print"   -> PRINT
     | "println" -> PRINTLN
     | "return"  -> RETURN
+    | "printStack"  -> PRINTSTACK
     | "true"    -> CSTBOOL 1
     | "void"    -> VOID 
     | "while"   -> WHILE         

--- a/2022-jan/opgave-2/MicroC/CPar.fsy
+++ b/2022-jan/opgave-2/MicroC/CPar.fsy
@@ -15,6 +15,7 @@ let nl = CstI 10
 %token <string> CSTSTRING NAME
 
 %token CHAR ELSE IF INT NULL PRINT PRINTLN RETURN VOID WHILE
+%token PRINTSTACK
 %token PLUS MINUS TIMES DIV MOD
 %token EQ NE GT LT GE LE
 %token NOT SEQOR SEQAND
@@ -95,6 +96,7 @@ Stmt:
 
 StmtM:  /* No unbalanced if-else */
     Expr SEMI                           { Expr($1)             }
+  | PRINTSTACK Expr SEMI                { PrintStack $2        }
   | RETURN SEMI                         { Return None          }
   | RETURN Expr SEMI                    { Return(Some($2))     }
   | Block                               { $1                   }
\end{minted}

Ved at parse \texttt{fac.c}, får man følgende abstrakte syntakstræ, hvor \texttt{printStack} ses på linje 25 og 33:

\bgroup
\footnotesize
\begin{minted}[linenos]{fsharp}
> open ParseAndRun;;
> fromFile "fac.c";;
val it : Absyn.program =
  Prog
    [Vardec (TypI, "nFac"); Vardec (TypI, "resFac");
     Fundec
       (None, "main", [(TypI, "n")],
        Block
          [Dec (TypI, "i"); Stmt (Expr (Assign (AccVar "i", CstI 0)));
           Stmt (Expr (Assign (AccVar "nFac", CstI 0)));
           Stmt
             (While
                (Prim2 ("<", Access (AccVar "i"), Access (AccVar "n")),
                 Block
                   [Stmt
                      (Expr
                         (Assign
                            (AccVar "resFac",
                             Call ("fac", [Access (AccVar "i")]))));
                    Stmt
                      (Expr
                         (Assign
                            (AccVar "i",
                             Prim2 ("+", Access (AccVar "i"), CstI 1))))]));
           Stmt (PrintStack (CstI 42))]);
     Fundec
       (Some TypI, "fac", [(TypI, "n")],
        Block
          [Stmt
             (Expr
                (Assign
                   (AccVar "nFac", Prim2 ("+", Access (AccVar "nFac"), CstI 1))));
           Stmt (PrintStack (Access (AccVar "nFac")));
           Stmt
             (If
                (Prim2 ("==", Access (AccVar "n"), CstI 0),
                 Return (Some (CstI 1)),
                 Return
                   (Some
                      (Prim2
                         ("*", Access (AccVar "n"),
                          Call
                            ("fac", [Prim2 ("-", Access (AccVar "n"), CstI 1)]))))))])]
\end{minted}
\egroup

\subsection{\texttt{Machine.fs}}

\begin{minted}{diff}
--- a/2022-jan/opgave-2/MicroC/Machine.fs
+++ b/2022-jan/opgave-2/MicroC/Machine.fs
@@ -14,6 +14,7 @@ module Machine
 type label = string
 
 type instr =
+  | PRINTSTACK
   | Label of label             (* symbolic label; pseudo-instruc. *)
   | CSTI of int                (* constant                        *)
   | ADD                        (* addition                        *)
@@ -91,6 +92,7 @@ let CODEPRINTI = 22
 let CODEPRINTC = 23
 let CODELDARGS = 24
 let CODESTOP   = 25;
+let CODEPRINTSTACK = 26
 
 (* Bytecode emission, first pass: build environment that maps 
    each label to an integer address in the bytecode.
@@ -98,6 +100,7 @@ let CODESTOP   = 25;
 
 let makelabenv (addr, labenv) instr = 
     match instr with
+    | PRINTSTACK     -> (addr+1, labenv)
     | Label lab      -> (addr, (lab, addr) :: labenv)
     | CSTI i         -> (addr+2, labenv)
     | ADD            -> (addr+1, labenv)
@@ -130,6 +133,7 @@ let makelabenv (addr, labenv) instr =
 
 let rec emitints getlab instr ints = 
     match instr with
+    | PRINTSTACK     -> CODEPRINTSTACK :: ints
     | Label lab      -> ints
     | CSTI i         -> CODECSTI   :: i :: ints
     | ADD            -> CODEADD    :: ints
\end{minted}

\subsection{\texttt{Machine.java}}

Forneden ses kodeændringerne i \texttt{Machine.java} efterfulgt af en forklaring.

\begin{minted}[linenos]{diff}
--- a/2022-jan/opgave-2/MicroC/Machine.java
+++ b/2022-jan/opgave-2/MicroC/Machine.java
@@ -35,7 +35,8 @@ 
     LDARGS = 24,
-    STOP = 25;
+    STOP = 25,
+    PRINTSTACK = 26;
 
@@ -127,6 +128,10 @@
       case PRINTC:
         System.out.print((char)(s[sp])); break; 
+      case PRINTSTACK:
+        int v = s[sp--];
+        printStack(v, s, bp, sp);
+        break;
       case LDARGS:
@@ -140,6 +145,31 @@
   }
 
+  static void printStack(int v, int s[], int bp, int sp) {
+    System.out.format("-Print Stack %d----------------%n", v);
+
+    // Iterate over all stack frames
+    while (bp != -999) {
+      // Print current stack frame
+      System.out.println("Stack Frame");
+      for (int i = sp; i >= bp; i--) {
+        System.out.format(" s[%d]: Local/Temp = %d%n", i, s[i]);
+      }
+      System.out.format(" s[%d]: bp = %d%n", bp - 1, s[bp - 1]);
+      System.out.format(" s[%d]: ret = %d%n", bp - 2, s[bp - 2]);
+
+      // Move to next stack frame
+      sp = bp - 3;
+      bp = s[bp - 1];
+    }
+
+    // Print remaining items on the stack: global variables
+    System.out.println("Global");
+    for (int i = sp; i >= 0; i--) {
+      System.out.format(" s[%d]: %d%n", i, s[i]);
+    }
+  }
+
   // Print the stack machine instruction at p[pc]
 
   static String insname(int[] p, int pc) {
@@ -170,6 +200,7 @@ class Machine {
     case STOP:   return "STOP";
+    case PRINTSTACK: return "PRINTSTACK";
     default:     return "<unknown>";
     }
   }
\end{minted}

Linje 7 viser, at bytekoden 26 forbindes med operationen \texttt{PRINTSTACK}.
\\\\
Linje 12-15 håndterer \texttt{PRINTSTACK}-instruktionen ved at poppe den øverste værdi af stakken (l. 13) og kalde en ny funktion \texttt{printStack} med denne værdi samt den nuværende tilstand af stakken.
\\\\
Linje 20-43 viser koden for \texttt{printStack}-funktionen:
\begin{enumerate}
    \item Først printes den poppede værdi (l. 21)
    \item Der itereres over alle aktiveringsposter (l. 24). Det gøres ved at følge alle basepegere (l. 34-35), indtil basepegeren bliver \texttt{-999}, da vi på det tidspunkt er nået til \texttt{main}-funktionens aktiveringspost.
    \begin{enumerate}
        \item Print aktiveringsposten ved at gennemgå alle adresser mellem basepegeren og stakpegeren.
        \item Print derefter \texttt{ret} og \texttt{bp} manuelt. Basepegeren peger på første lokale værdi i aktiveringsposten, og derfor må den gamle basepeger være på adressen lige før den nuværende basepeger. Og \texttt{ret} findes på adressen før den gamle basepeger.
        \item Opdater basepegeren og stakpegeren, så vi kan læse den foregående aktiveringspost. Fordi de to adresser lige under den nuværende basepeger er allokeret til \texttt{ret} og den gamle basepeger, kan vi sætte den næste stakpeger til adressen 3 lokationer under basepegeren --- altså toppen af den næste aktiveringspost.
    \end{enumerate}
    \item Hvis der er defineret globale variable, findes de på adresserne mellem 0 og starten af \texttt{main}-funktionen. Derfor printer vi alle adresser efter den foregående iteration som globale variable.
\end{enumerate}

\subsection{\texttt{Comp.fs}}

Ændringen ses her:

\begin{minted}{diff}
--- a/2022-jan/opgave-2/MicroC/Comp.fs
+++ b/2022-jan/opgave-2/MicroC/Comp.fs
@@ -144,6 +144,8 @@ let rec cStmt stmt (varEnv : varEnv) (funEnv : funEnv) : instr list =
       [RET (snd varEnv - 1)]
     | Return (Some e) -> 
       cExpr e varEnv funEnv @ [RET (snd varEnv)]
+    | PrintStack e ->
+      cExpr e varEnv funEnv @ [PRINTSTACK]
 
 and cStmtOrDec stmtOrDec (varEnv : varEnv) (funEnv : funEnv) : varEnv * instr list = 
     match stmtOrDec with 
\end{minted}

Først kompileres udtrykket, der bruges som argument til \texttt{printStack}. Når dette er kompileret, vil udtrykkets endelige værdi findes på toppen af stakken ved køretid. Derfor mangler der blot selve \texttt{PRINTSTACK}-instruktionen, der popper værdien af stakken og bruger den, hvorfor den er tilføjet til sidst.
\\\\
Bytekoden for \texttt{fac.c} ses forneden. Linje 10 og 12 viser instruktionerne til at printe stakken: linje 10 bruger den konstante værdi 42, og linje 12 indlæser værdien på adresse 0 (den globale variabel \texttt{nFac}).

\begin{minted}[fontsize=\small, linenos]{fsharp}
> open ParseAndComp;;
> compile "fac";;
val it : Machine.instr list =
  [INCSP 1; INCSP 1; LDARGS; CALL (1, "L1"); STOP; Label "L1"; INCSP 1; GETBP;
   CSTI 1; ADD; CSTI 0; STI; INCSP -1; CSTI 0; CSTI 0; STI; INCSP -1;
   GOTO "L4"; Label "L3"; CSTI 1; GETBP; CSTI 1; ADD; LDI; CALL (1, "L2"); STI;
   INCSP -1; GETBP; CSTI 1; ADD; GETBP; CSTI 1; ADD; LDI; CSTI 1; ADD; STI;
   INCSP -1; INCSP 0; Label "L4"; GETBP; CSTI 1; ADD; LDI; GETBP; CSTI 0; ADD;
   LDI; LT; IFNZRO "L3";
   CSTI 42; PRINTSTACK;
   INCSP -1; RET 0; Label "L2"; CSTI 0; CSTI 0; LDI; CSTI 1; ADD; STI; INCSP -1;
   CSTI 0; LDI; PRINTSTACK;
   GETBP; CSTI 0; ADD; LDI; CSTI 0; EQ; IFZERO "L5"; CSTI 1; RET 1; GOTO "L6";
   Label "L5"; GETBP; CSTI 0; ADD; LDI; GETBP; CSTI 0; ADD; LDI; CSTI 1; SUB;
   CALL (1, "L2"); MUL; RET 1; Label "L6"; INCSP 0; RET 0]
\end{minted}

Ved kørsel af \texttt{fac.c} gennem bytekodefortolkeren, fås følgende output:

\begin{minted}[fontsize=\small]{text}
$ java Machine.java fac.out 1
-Print Stack 1----------------
Stack Frame
 s[9]: Local/Temp = 0
 s[8]: bp = 4
 s[7]: ret = 39
Stack Frame
 s[6]: Local/Temp = 1
 s[5]: Local/Temp = 0
 s[4]: Local/Temp = 1
 s[3]: bp = -999
 s[2]: ret = 8
Global
 s[1]: 0
 s[0]: 1
-Print Stack 42----------------
Stack Frame
 s[5]: Local/Temp = 1
 s[4]: Local/Temp = 1
 s[3]: bp = -999
 s[2]: ret = 8
Global
 s[1]: 1
 s[0]: 1

Ran 0.005 seconds
\end{minted}

\subsection{Gennemgang af bytekode}

\begin{minted}{asm}
INCSP 1;       // nFac som global variabel
INCSP 1;       // resFac som global variabel
LDARGS;        // Loade parameter n fra kommandolinie
CALL (1,"L1"); // Kalde main med n som argument.
STOP;          // Stop ved retur fra main.
Label "L1";                                    // Main
    INCSP 1;                                   // i som lokal variabel
    GETBP; CSTI 1; ADD; CSTI 0; STI; INCSP -1; // i = 0
    CSTI 0; CSTI 0; STI; INCSP -1;             // Sætte nFac = 0
    GOTO "L4";                                 // Hop til while-løkken
Label "L3";                       // While-løkkens krop
    CSTI 1;                       // Adresse på resFac
    GETBP; CSTI 1; ADD; LDI;      // Læg i på stakken som argument
    CALL (1,"L2"); STI; INCSP -1; // Kald fac(i) og gem resultatet i resFac (hvis adresse er på toppen af stakken)
    GETBP; CSTI 1; ADD;           // Beregn adressen på i
    GETBP; CSTI 1; ADD; LDI;      // Læs værdien af i
    CSTI 1; ADD; STI; INCSP -1;   // Gem i + 1 på i's adresse
    INCSP 0;                      // Ligegyldig oprydning pga. blokkens slutning
Label "L4";                  // While-løkken
    GETBP; CSTI 1; ADD; LDI; // Læs værdien af i
    GETBP; CSTI 0; ADD; LDI; // Læs værdien af n
    LT; IFNZRO "L3";         // Hvis i < n, gå ind i løkkens krop
    CSTI 42; PRINTSTACK;     // Når while-løkken er færdig, kaldes printStack med 42 som argument
    INCSP -1;                // Oprydning på stakken (gammel basepeger deallokeres)
    RET 0;                   // Returnér
Label "L2";                      // Label til fac(int n)
    CSTI 0; CSTI 0; LDI; CSTI 1; // Adresse på nFac, værdi af nFac, og 1 lægges alle på stakken
    ADD; STI; INCSP -1;          // nFac = nFac + 1
    CSTI 0; LDI; PRINTSTACK;     // Kør printStack med nFac som argument
    GETBP; CSTI 0; ADD; LDI;     // Læs første argument: n
    CSTI 0; EQ; IFZERO "L5";     // Hvis n != 0, gå til L5
    CSTI 1; RET 1; GOTO "L6";    // Hvis n = 0, returnér 1
Label "L5";                               // False-case i fac
    GETBP; CSTI 0; ADD; LDI;              // Læs værdien af n
    GETBP; CSTI 0; ADD; LDI; CSTI 1; SUB; // Beregn n - 1
    CALL (1,"L2");                        // Kald fac med n - 1 som argument
    MUL; RET 1;                           // Gang resultat med n og returnér
Label "L6";        // Ubrugt kode til true-case i fac
    INCSP 0; RET 0 // Ubrugt
\end{minted}

\subsection{Forståelse af uddata}

\begin{minted}{c}
MicroC % java Machine fac.out 1
-Print Stack 1----------------
Stack Frame // Funktion fac
s[9]: Local/Temp = 0 // Stakplads til lokal variabel n
s[8]: bp = 4
s[7]: ret = 39
Stack Frame // Main
s[6]: Local/Temp = 1 // Temporær værdi: adressen på resFac, hvortil resultatet af fac(i) skal skrives
s[5]: Local/Temp = 0 // Lokal variabel i
s[4]: Local/Temp = 1 // Lokal variabel n
s[3]: bp = -999
s[2]: ret = 8
Global
s[1]: 0 // resFac
s[0]: 1 // nFac
-Print Stack 42----------------
Stack Frame // Main
s[5]: Local/Temp = 1 // Lokal variabel i
s[4]: Local/Temp = 1 // Lokal variabel n
s[3]: bp = -999
s[2]: ret = 8
Global
s[1]: 1 // resFac
s[0]: 1 // nFac
\end{minted}
